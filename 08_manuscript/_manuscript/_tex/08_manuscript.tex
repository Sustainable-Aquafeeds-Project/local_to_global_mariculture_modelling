% Options for packages loaded elsewhere
\PassOptionsToPackage{unicode}{hyperref}
\PassOptionsToPackage{hyphens}{url}
\PassOptionsToPackage{dvipsnames,svgnames,x11names}{xcolor}
%
\documentclass[
  a4paper,
]{article}

\usepackage{amsmath,amssymb}
\usepackage{setspace}
\usepackage{iftex}
\ifPDFTeX
  \usepackage[T1]{fontenc}
  \usepackage[utf8]{inputenc}
  \usepackage{textcomp} % provide euro and other symbols
\else % if luatex or xetex
  \usepackage{unicode-math}
  \defaultfontfeatures{Scale=MatchLowercase}
  \defaultfontfeatures[\rmfamily]{Ligatures=TeX,Scale=1}
\fi
\usepackage{lmodern}
\ifPDFTeX\else  
    % xetex/luatex font selection
\fi
% Use upquote if available, for straight quotes in verbatim environments
\IfFileExists{upquote.sty}{\usepackage{upquote}}{}
\IfFileExists{microtype.sty}{% use microtype if available
  \usepackage[]{microtype}
  \UseMicrotypeSet[protrusion]{basicmath} % disable protrusion for tt fonts
}{}
\makeatletter
\@ifundefined{KOMAClassName}{% if non-KOMA class
  \IfFileExists{parskip.sty}{%
    \usepackage{parskip}
  }{% else
    \setlength{\parindent}{0pt}
    \setlength{\parskip}{6pt plus 2pt minus 1pt}}
}{% if KOMA class
  \KOMAoptions{parskip=half}}
\makeatother
\usepackage{xcolor}
\setlength{\emergencystretch}{3em} % prevent overfull lines
\setcounter{secnumdepth}{-\maxdimen} % remove section numbering
% Make \paragraph and \subparagraph free-standing
\makeatletter
\ifx\paragraph\undefined\else
  \let\oldparagraph\paragraph
  \renewcommand{\paragraph}{
    \@ifstar
      \xxxParagraphStar
      \xxxParagraphNoStar
  }
  \newcommand{\xxxParagraphStar}[1]{\oldparagraph*{#1}\mbox{}}
  \newcommand{\xxxParagraphNoStar}[1]{\oldparagraph{#1}\mbox{}}
\fi
\ifx\subparagraph\undefined\else
  \let\oldsubparagraph\subparagraph
  \renewcommand{\subparagraph}{
    \@ifstar
      \xxxSubParagraphStar
      \xxxSubParagraphNoStar
  }
  \newcommand{\xxxSubParagraphStar}[1]{\oldsubparagraph*{#1}\mbox{}}
  \newcommand{\xxxSubParagraphNoStar}[1]{\oldsubparagraph{#1}\mbox{}}
\fi
\makeatother


\providecommand{\tightlist}{%
  \setlength{\itemsep}{0pt}\setlength{\parskip}{0pt}}\usepackage{longtable,booktabs,array}
\usepackage{calc} % for calculating minipage widths
% Correct order of tables after \paragraph or \subparagraph
\usepackage{etoolbox}
\makeatletter
\patchcmd\longtable{\par}{\if@noskipsec\mbox{}\fi\par}{}{}
\makeatother
% Allow footnotes in longtable head/foot
\IfFileExists{footnotehyper.sty}{\usepackage{footnotehyper}}{\usepackage{footnote}}
\makesavenoteenv{longtable}
\usepackage{graphicx}
\makeatletter
\def\maxwidth{\ifdim\Gin@nat@width>\linewidth\linewidth\else\Gin@nat@width\fi}
\def\maxheight{\ifdim\Gin@nat@height>\textheight\textheight\else\Gin@nat@height\fi}
\makeatother
% Scale images if necessary, so that they will not overflow the page
% margins by default, and it is still possible to overwrite the defaults
% using explicit options in \includegraphics[width, height, ...]{}
\setkeys{Gin}{width=\maxwidth,height=\maxheight,keepaspectratio}
% Set default figure placement to htbp
\makeatletter
\def\fps@figure{htbp}
\makeatother
% definitions for citeproc citations
\NewDocumentCommand\citeproctext{}{}
\NewDocumentCommand\citeproc{mm}{%
  \begingroup\def\citeproctext{#2}\cite{#1}\endgroup}
\makeatletter
 % allow citations to break across lines
 \let\@cite@ofmt\@firstofone
 % avoid brackets around text for \cite:
 \def\@biblabel#1{}
 \def\@cite#1#2{{#1\if@tempswa , #2\fi}}
\makeatother
\newlength{\cslhangindent}
\setlength{\cslhangindent}{1.5em}
\newlength{\csllabelwidth}
\setlength{\csllabelwidth}{3em}
\newenvironment{CSLReferences}[2] % #1 hanging-indent, #2 entry-spacing
 {\begin{list}{}{%
  \setlength{\itemindent}{0pt}
  \setlength{\leftmargin}{0pt}
  \setlength{\parsep}{0pt}
  % turn on hanging indent if param 1 is 1
  \ifodd #1
   \setlength{\leftmargin}{\cslhangindent}
   \setlength{\itemindent}{-1\cslhangindent}
  \fi
  % set entry spacing
  \setlength{\itemsep}{#2\baselineskip}}}
 {\end{list}}
\usepackage{calc}
\newcommand{\CSLBlock}[1]{\hfill\break\parbox[t]{\linewidth}{\strut\ignorespaces#1\strut}}
\newcommand{\CSLLeftMargin}[1]{\parbox[t]{\csllabelwidth}{\strut#1\strut}}
\newcommand{\CSLRightInline}[1]{\parbox[t]{\linewidth - \csllabelwidth}{\strut#1\strut}}
\newcommand{\CSLIndent}[1]{\hspace{\cslhangindent}#1}

\usepackage{lineno}\linenumbers
\usepackage[scale=0.75]{geometry}
\usepackage[noblocks]{authblk}
\usepackage{lscape}
\usepackage[para]{footmisc}
\usepackage{typearea}

\renewcommand*{\Authsep}{, }
\renewcommand*{\Authand}{, }
\renewcommand*{\Authands}{, }
\renewcommand\Affilfont{\small}
\makeatletter
\@ifpackageloaded{caption}{}{\usepackage{caption}}
\AtBeginDocument{%
\ifdefined\contentsname
  \renewcommand*\contentsname{Table of contents}
\else
  \newcommand\contentsname{Table of contents}
\fi
\ifdefined\listfigurename
  \renewcommand*\listfigurename{List of Figures}
\else
  \newcommand\listfigurename{List of Figures}
\fi
\ifdefined\listtablename
  \renewcommand*\listtablename{List of Tables}
\else
  \newcommand\listtablename{List of Tables}
\fi
\ifdefined\figurename
  \renewcommand*\figurename{Figure}
\else
  \newcommand\figurename{Figure}
\fi
\ifdefined\tablename
  \renewcommand*\tablename{Table}
\else
  \newcommand\tablename{Table}
\fi
}
\@ifpackageloaded{float}{}{\usepackage{float}}
\floatstyle{ruled}
\@ifundefined{c@chapter}{\newfloat{codelisting}{h}{lop}}{\newfloat{codelisting}{h}{lop}[chapter]}
\floatname{codelisting}{Listing}
\newcommand*\listoflistings{\listof{codelisting}{List of Listings}}
\makeatother
\makeatletter
\makeatother
\makeatletter
\@ifpackageloaded{caption}{}{\usepackage{caption}}
\@ifpackageloaded{subcaption}{}{\usepackage{subcaption}}
\makeatother

\ifLuaTeX
  \usepackage{selnolig}  % disable illegal ligatures
\fi
\usepackage{bookmark}

\IfFileExists{xurl.sty}{\usepackage{xurl}}{} % add URL line breaks if available
\urlstyle{same} % disable monospaced font for URLs
\hypersetup{
  pdftitle={Local to global N input modelling},
  pdfauthor={Tormey Reimer; Richard S. Cottrell; Alexandra Johne; Sowdamini Sesha Prasad; Marceau Cormery; Gage Clawson; Scott Hadley; Helen Hamilton; Benjamin S. Halpern; Catriona Macleod; Camille White; Julia L. Blanchard},
  colorlinks=true,
  linkcolor={blue},
  filecolor={Maroon},
  citecolor={Blue},
  urlcolor={Blue},
  pdfcreator={LaTeX via pandoc}}


\title{Local to global N input modelling}


\author[12]{Tormey Reimer}
\author{Richard S. Cottrell}
\author[3]{Alexandra Johne}
\author{Sowdamini Sesha Prasad}
\author{Marceau Cormery}
\author{Gage Clawson}
\author{Scott Hadley}
\author{Helen Hamilton}
\author{Benjamin S. Halpern}
\author{Catriona Macleod}
\author{Camille White}
\author{Julia L. Blanchard}

\affil[1]{Institute for Marine and Antarctic Studies}
\affil[2]{Centre for Marine Socioecology}
\affil[1]{Institute for Marine and Antarctic Studies}


\date{}

\begin{document}
\maketitle


\setstretch{1.15}
\section{Introduction}\label{introduction}

Aquaculture is now the dominant form of aquatic animal food (herein
`seafood') production and is expected to be the primary way we meet
future seafood demand. Freshwater systems will likely continue to
provide the majority of farmed seafood but marine aquaculture is also
poised to expand substantially in numerous areas. Farmed marine fish and
invertebrates are produced near exclusively in coastal waters, and
nearly three quarters of this production is dependent on human-made
feeds. Nearshore locations and feed inputs are necessary to maintain
profitable and productive farming operations but coastal aquaculture
generates a number of challenges. In the crowded coastal zone,
aquaculture operations can conflict with other stakeholder uses such as
recreation, fishing, renewable energy, transport, and tourism. And while
farming marine fish typically generates a far smaller nutrient footprint
than livestock farming, the overt nature of aquaculture in nearshore
regions and evidence of localised nutrient impacts around fish farms
remains a primary public and scientific concern. Identifying strategies
that reduce ecosystem impacts from fish farm waste therefore represents
an important goal for improving marine aquaculture sustainability and
maintaining the sector's social licence to operate.

Aquaculture feeds represent an important lever for reducing nutrient
waste impacts around fish farms. Like all farmed animals, fish and
invertebrates must digest the nutrients contained in feeds before they
can be used for growth. Any nutrients left undigested are egested as
solid waste, and dissolved wastes are excreted as metabolic waste
products. Further, some feed inevitably remains uneaten and is lost to
the surrounding ecosystem. Particulate organic matter (both feed and
faeces) that settles can simplify benthic communities as the oxygen
demand from its decomposition drives the production of sulphides that
kill less mobile faunal, encouraging a lower diversity of opportunistic
scavengers and the growth of bacterial mats (e.g., Beggiatoa spp). Thus,
the chemical composition of the ingredients used in aquaculture feeds
and their digestibility for the farmed species has significant
implications for the nature and reactivity of the waste generate by
marine aquaculture.

Firstly the overall volume of nutrient waste is dictated by the nature
and intensity of production, that is the farm size, the density of
farmed animals and the feed requirements and efficiency of the species
grown. SecondlDeposition of waste is heavily influenced by water depth
and current speed at the farming site. Once

As farmed fish and invertebrates are fed, whatever

Nutand its impact on marine ecosystems is influenced by many factors. *
Farm size * Depth * Current speed * Benthic impact - sediment
type/faunal assemblages/wider marine community * High turnover
environments - nitrogen enriched areas * Feed influences all of these
things

The primary source of organic waste from fed aquaculture production
comes from the excretion and faeces of the farmed animals and through
uneaten feed that dissolves in the water column or settles on the
benthos. The nature and impact of this waste are influenced heavily by
the composition of the feeds fed to farmed animals.

\subsection{P2}\label{p2}

\begin{itemize}
\tightlist
\item
  Waste from aquaculture farms and it's impact is influenced by many
  things but the composition of feeds plays a central role.
\item
  Waste from aquaculture farms has multiple sources.
\item
  The primary source of organic waste comes from the faeces and
  excretion of the fish or invertebrates.
\item
  Uneaten feed produced another key source.
\item
  The nature and impact of this waste are influenced heavily by the
  composition of the feeds fed to farmed animals
\item
  Many marine fish are naturally carnivorous so diets used to be high in
  fishmeal and oil but increasing fishmeal and oil prices along with
  concerns over the sustainability of marine ingredients have led to a
  reduction in their use across multiple farmed taxa
\item
  In lieu of fishmeal and oil, many plant-based ingredients such as soy
  protein concentrate, canola oil, and wheat gluten have replaced them.
\item
  Changes in feed composition influences the digestibility of the
  nutrients held in each feed and can alter the composition of waste.
\item
  Of particular concern are changes (increases) to the presence of
  reactive nitrogen and phosphorus in coastal waters that could have an
  effect on eutrophication.
\end{itemize}

\subsection{P3}\label{p3}

\begin{itemize}
\tightlist
\item
  Whether or not nutrients lead to eutrophication depends on the
  sensitivity of the receiving environment
\item
  Ecosystems that are already enriched through natural processes and
  whose biota is well adapted to substantial fluxes in available
  nutrients (e.g.~upwelling zones, dynamic coastal communities) may be
  less sensitive while oligotrophic ecosystem are likely to see
  considerable changes under nutrient enrichment scenarios.
\item
  To understand the impact of aquaculture waste under present day or
  future scenarios we need to quantify the volume, nature, and location
  of mariculture waste and determine the sensitivity of the receiving
  environments to that waste. Yet only recent estimates even give us the
  estimated location of marine farms let alone the volume of nature of
  the waste produced. To address this gap, we use existing maps of
  mariculture location with a bioenergtic model
\end{itemize}

\section{Statistical analysis}\label{statistical-analysis}

All analysis was conducted in R version 4.2 ``Pile of Leaves''
(\citeproc{ref-R_base}{R Core Team 2019}).

\section{Code availability}\label{code-availability}

This manuscript was written in Quarto (\citeproc{ref-quarto}{Allaire et
al. 2024}) using TinyTex (\citeproc{ref-tinytex}{\textbf{tinytex?}}) and
the acronyms extension (\citeproc{ref-acronyms}{Chaput 2024}). For a
full list of R packages used see the
\href{https://www.github.com}{lockfile}.

\phantomsection\label{refs}
\begin{CSLReferences}{1}{0}
\bibitem[\citeproctext]{ref-quarto}
Allaire, J. J., Charles Teague, Carlos Scheidegger, Yihui Xie, and
Christophe Dervieux. 2024. {``{Quarto}.''}
\url{https://doi.org/10.5281/zenodo.5960048}.

\bibitem[\citeproctext]{ref-acronyms}
Chaput, Remy. 2024. {``{acronyms}.''}
\url{https://github.com/rchaput/acronyms}.

\bibitem[\citeproctext]{ref-R_base}
R Core Team. 2019. \emph{R: A Language and Environment for Statistical
Computing}. Vienna, Austria: R Foundation for Statistical Computing.
\url{https://www.R-project.org}.

\end{CSLReferences}




\end{document}
